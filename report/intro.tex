% File: intro.tex
% Date: Sun Sep 01 09:45:52 2013 +0800
% Author: Yuxin Wu <ppwwyyxxc@gmail.com>

\section{简介}
本程序是一个基于光线追踪的3D渲染程序.
可以选用Phong模型\cite{phong}作为局部光照模型或Path Tracing\cite{pathtracing}作为全局光照模型,渲染3D场景.
支持平面、球、三角面片、三角网格(可从obj文件读入)几种几何对象,并可方便的扩展.
渲染效果上,支持软阴影,抗锯齿,景深,自定义纹理等功能, 并可对三角网格进行简化.
三角网格,全局渲染,网格简化均采用特定数据结构(KD树及堆)与多线程加速,效率很高.
同时,将渲染功能嵌入了图形界面,可以在图形界面上支持obj文件的预览及简化.

\subsection{依赖}
\begin{enumerate}
  \item 本程序用C++11编写,需要编译器支持C++11中的ranged loop, initializer list, type inference等语法,
    且需标准库包含\verb|std::shared_ptr, std::future|
    类.建议使用g++$ \ge$ 4.8 编译.

  \item OpenCV2 \footnote{\url{http://opencv.org}}

  \item ImageMagick \footnote{\url{http://www.imagemagick.org/script/index.php}}

\item Qt4  \footnote{\url{http://qt-project.org/}} (可选)
\end{enumerate}

\subsection{编译}
在\verb|src|目录中,使用\verb|make|命令和\verb|make gui|命令分别编译命令行程序与图形界面程序.

\subsection{使用}

\begin{enumerate}
    \item 命令行程序:
      直接运行,渲染一个演示场景. 可在\verb|main()|函数中选择不同的场景.

      程序调用opencv进行图像显示,显示时可通过键盘进行导航,导航方法见下表:
\begin{table}[H]
  \begin{tabular}{c|c}
    \shline
    \UArrow \DArrow& 屏幕以视点到其连线为轴旋转, \UArrow 顺时针\\
    \LArrow \RArrow & 围绕固定中心旋转视点\\
    \keystroke{h}\keystroke{j}\keystroke{k}\keystroke{l} & 视点及屏幕的平移 \\
    \keystroke{=}\keystroke{-} & 缩放\\
    \keystroke{>}\keystroke{<} & 固定视点旋转视角\\
    \keystroke{]}\keystroke{[} & 调节焦平面远近(景深模式下有用), \keystroke{]}调远\\
    \keystroke{s} & 保存当前图片至output.png\\
    \keystroke{p} & 输出当前视角信息\\
    \keystroke{q} \Esc & 退出\\
  \end{tabular}
  \centering
  \caption{场景导航快捷键\label{tab:navigate}}
\end{table}

      \item 图形界面程序:
        运行界面如\figref{gui}所示,
        运行后,通过open按钮选择obj文件,trace按钮渲染,smooth控制是否开启法向插值.
        其余按钮用于更改视角、渲染方式等参数,改变视角后重新trace即可生效.
        simplify按钮将obj模型按照给定的简化率进行简化,simplify rate表示简化后保留的面片所占比例.

\end{enumerate}
